


%%%% SHARED   %%%%%%%%%%%%%%%%%%%%%%%%%%%%%  

% for preventing newlines in answers (appear on a single line)
\newenvironment{hint}{\vspace{1.5mm}\par\noindent Hint:}{}
%
\makeatletter
\newcommand\gobblepars{%
    \@ifnextchar\par%
        {\expandafter\gobblepars\@gobble}%
        {}}
\makeatother
%%%%%%%%%%%%%%%%%%%%%%%%%%%%%%%%%%%%%%%%%%%%%%%




%%%% END-OF-CHAPTER  PROBLEMS   %%%%%%%%%%%%%%%%%%%%%%%%%%%%%  
\Newassociation{answer}{Answer}{answers}			% optional replacement for solution (if no space)
\Newassociation{solution}{Solution}{solutions}			% defining the three saved "streams", solutions --> back of 

% A special theorem style for end-of-chapter problems  --- P4.1, P4.2, P4.3, ...
\newtheoremstyle{problemstyle}
  {0.3\topsep}   % Space above
  {0.7\topsep}   % Space below
  {\normalfont}  % Body font
  {0pt}       % Indent amount (empty value is the same as 0pt)
  {\bfseries} % Theorem head font
  {}          % Punctuation after theorem head
  {6pt plus 1pt minus 1pt} % Space after theorem head
  {P\thmnumber{#2} \thmnote{ (#3)}} % Theorem head spec
%
%
\theoremstyle{problemstyle}
\newtheorem{problem}{}[]

% a container for end-of-chapter exercises and end-of-chapter problems 
\newenvironment{problems}[1]{
	\setcounter{problem}{0}
	\Opensolutionfile{answers}[99anssol/answers_#1]
	\Opensolutionfile{solutions}[99anssol/solutions_#1] 
}{
	\Closesolutionfile{answers}
	\Closesolutionfile{solutions}
}
% Problem answers and solution display
%
\newcommand{\showProblemAnswers}[1]{	
	%	\renewcommand{\Answerlabel}[1]{E##1.}
	\renewenvironment{Answer}[1]{\textbf{P##1}}{\gobblepars}
	\input{99anssol/answers_#1}
}
%
\newcommand{\showProblemSolutions}[1]{	
	\renewcommand{\Solutionlabel}[1]{\textbf{P##1}}
	\input{99anssol/solutions_#1}
}
%%%%%%%%%%%%%%%%%%%%%%%%%%%%%%%%%%%%%%%%%%%%%%%





%%%% END-OF-SECTION  EXERCISES   %%%%%%%%%%%%%%%%%%%%%%%%%%%%%
\Newassociation{esolution}{ESolution}{esolutions}			% defining the three saved "streams", solutions --> back of book
\Newassociation{eanswer}{EAnswer}{eanswers}			% optional replacement for solution (if no space)


% A special theorem style for end-of-section exercises  --- E1.1, E1.2, E1.3, ...
\newtheoremstyle{exercisestyle}
  {0.3\topsep}   % Space above
  {0.7\topsep}   % Space below
  {\normalfont}  % Body font
  {0pt}       % Indent amount (empty value is the same as 0pt)
  {\bfseries} % Theorem head font
  {}          % Punctuation after theorem head
  {5pt plus 1pt minus 1pt} % Space after theorem head
  {E\thmnumber{#2}\thmnote{ (#3)}} % Theorem head spec
%
%  
\theoremstyle{exercisestyle}
\newtheorem{exercise}{}[]

	
% a container for end-of-section exercises and end-of-chapter problems 
\newcommand{\beforeFirstExercises}[1]{
	\setcounter{exercise}{0}
	\Opensolutionfile{esolutions}[99anssol/esolutions_#1] 
	\Opensolutionfile{eanswers}[99anssol/eanswers_#1]
}
\newcommand{\afterLastExercises}[1]{
	\Closesolutionfile{esolutions}
	\Closesolutionfile{eanswers}
}
\newenvironment{exercises}[1]{ }{ }	% empty exercises env...
% Exercises answers, and solution display
\newcommand{\showExerciseSolutions}[1]{	
	\renewcommand{\ESolutionlabel}[1]{\textbf{E##1}}
	\input{99anssol/esolutions_#1}
}
%
\newcommand{\showExerciseAnswers}[1]{	
	%	\renewcommand{\Answerlabel}[1]{E##1.}
	\renewenvironment{EAnswer}[1]{\textbf{E##1}}{\gobblepars}
	\input{99anssol/eanswers_#1}
}
%%%%%%%%%%%%%%%%%%%%%%%%%%%%%%%%%%%%%%%%%%%%%%%






%%%%% COLORS used in cerain figs

\definecolor{captioncolor}{HTML}{005587}
\definecolor{linecolor}{HTML}{0074C8}
\definecolor{fillcolor}{cmyk}{0.1,0.05,0,0}
\definecolor{brickcolor}{HTML}{F0D8B2}
\definecolor{blockcolor}{HTML}{B6B6B6}
\definecolor{groundcolor}{HTML}{E4D8C5}
\definecolor{earthcolor}{HTML}{C5FFFF}
\definecolor{watercolor}{cmyk}{0.1,0.05,0,0}
\definecolor{codecolor}{HTML}{FFF7E0}
\definecolor{linenumcolor}{HTML}{8C8C8C}
%\definecolor{headercolor}{HTML}{F5F5F5}
\definecolor{headercolor}{HTML}{E9EFEF}






%%%%   MULTI-PART EXERCISES AND PROBLEMS  %%%%%%%%%%%%%%%%%%%%%%%%%%%
%--------exparts and exparts*
% For exercises with sub-parts. Use exparts* to print them across:
% (a) xxx  (b) yyy  (c) zzz ..
% while exparts prints them down:
% (a) xxx
% (b) yyy
% Use this way:
% \begin{exparts}
% 	\partitem        % not just \item so works in *-ed version.
% \end{exparts}
%
\newcounter{expartscount}
\newenvironment{exparts}{%
\def\partsitem{\item\relax}%
\begin{expartslist}%
}{%
\end{expartslist}}
%
\newenvironment{exparts*}{%
\def\partsitem{\penalty-400\hskip1em\relax\hbox{%
\refstepcounter{expartscount}\textbf{\alph{expartscount})}\hspace*{.1em}}\nobreak}
\begin{expartslist}
\setlength{\itemindent}{-1em}
\rightskip=0pt plus7em\spaceskip=.3333em\xspaceskip=.5em\relax%
\item[]}{%
\end{expartslist}}
%
%
\newenvironment{expartslist}{%
\begin{list}{\textbf{(\alph{expartscount})}}{
  \usecounter{expartscount}
  % \setlength{\labelindent}{5em}	% not to be specified: computed form next three
  \setlength{\leftmargin}{2em}		% = lebal
  \setlength{\labelwidth}{2em}
  \setlength{\labelsep}{0.5em}
  \setlength{\itemindent}{0em}
  \setlength{\listparindent}{\parindent}		% if multiple paragraphs in list item
  \setlength{\rightmargin}{0em}
  % VERTICAL
  \setlength{\topsep}{0.4em}
  \setlength{\parskip}{0ex}
  \setlength{\partopsep}{0ex}
  \setlength{\parsep}{0ex}
  \setlength{\itemsep}{0ex}
  }%
}{%
\end{list}}
%
%...........end: exparts and exparts*


%--------ansparts* --- for inline ansers
\usepackage[inline]{enumitem}
\newenvironment{ansparts*}{%
\def\partsitem{\item}
\begin{enumerate*}[label=\textbf{(\alph*)}]}{
\end{enumerate*}}

%--------solparts --- for paragraph solutions
\newenvironment{solparts}{%
\def\partsitem{\item\relax}%
\begin{solpartslist}%
}{%
\end{solpartslist}}
%
\newcounter{solpartscount}
\newenvironment{solpartslist}{%
\begin{list}{\textbf{(\alph{solpartscount})}}{
  \usecounter{solpartscount}
  % \setlength{\labelindent}{5em}	% not to be specified: computed form next three
  \setlength{\leftmargin}{0em}
  \setlength{\labelwidth}{3em}
  \setlength{\labelsep}{0.5em}
  \setlength{\itemindent}{1.8em}
  \setlength{\listparindent}{\parindent}		% if multiple paragraphs in list item
  \setlength{\rightmargin}{0em}
  % VERTICAL
  \setlength{\topsep}{0.1em}
  \setlength{\parskip}{0ex}
  \setlength{\partopsep}{0ex}
  \setlength{\parsep}{0ex}
  \setlength{\itemsep}{0ex}
  }%
}{%
\end{list}}
%%%%%%%%%%%%%%%%%%%%%%%%%%%%%%%%%%%%%%%%%%%%%%%%%%%%%%%%%%%


